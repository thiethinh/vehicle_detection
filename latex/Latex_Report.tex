\documentclass[conference]{IEEEtran}
% Cấu hình gói ngôn ngữ và font chữ
\usepackage[utf8]{inputenc}
\usepackage[T5]{fontenc}
\usepackage[vietnamese]{babel}

% Các lib cho toán học, hình ảnh, bảng biểu
\usepackage{cite}
\usepackage{amsmath,amssymb,amsfonts}
\usepackage{algorithmic}
\usepackage{graphicx}
\usepackage{textcomp}
\usepackage{xcolor}
\usepackage{booktabs} % Để tạo bảng đẹp hơn
\usepackage{float}
\usepackage{hyperref}

% Định nghĩa lại tên các phần tự động của IEEEtran sang tiếng Việt
\renewcommand{\IEEEkeywordsname}{Từ khóa}

% Hàm giả lập hình ảnh (để code chạy được ngay cả khi bạn chưa có file ảnh)
% Khi có ảnh thật, bạn hãy thay lệnh \placeholderimage bằng \includegraphics
\newcommand{\placeholderimage}[2]{
    \begin{figure}[htbp]
        \centering
        \framebox{
            \begin{minipage}{0.9\linewidth}
                \centering
                \vspace{1.5cm}
                \textbf{VỊ TRÍ CHÈN ẢNH}\\
                #1
                \vspace{1.5cm}
            \end{minipage}
        }
        \caption{#2}
        \label{#1}
    \end{figure}
}
    
\begin{document}

\title{Ứng dụng trí tuệ nhân tạo trong việc nhận diện và phân loại phương tiện giao thông từ Camera }

\author{\IEEEauthorblockN{1\textsuperscript{st} Nguyễn Thiết Hinh}
\IEEEauthorblockA{\textit{Trường ĐH Nông Lâm TP.HCM} \\
\textit{Khoa: Công nghệ thông tin}\\
\textit{Email: 23130110@st.hcmuaf.edu.vn}}
\and
\IEEEauthorblockN{2\textsuperscript{nd} Nguyễn Hoàng Thái}
\IEEEauthorblockA{\textit{Trường ĐH Nông Lâm TP.HCM} \\
\textit{Khoa: Công nghệ thông tin}\\
\textit{Email: 23130296@st.hcmuaf.edu.vn}}
\and
\IEEEauthorblockN{3\textsuperscript{rd} Tô Tấn Phát}
\IEEEauthorblockA{\textit{Trường ĐH Nông Lâm TP.HCM} \\
\textit{Khoa: Công nghệ thông tin}\\
\textit{Email: 23130232@st.hcmuaf.edu.vn}}
}

\maketitle
% ------ Đánh dấu số trang ----
\thispagestyle{plain} 
\pagestyle{plain}


\begin{abstract}
Trong kỷ nguyên đô thị hóa hiện đại, việc tối ưu hóa Hệ thống Giao thông Thông minh (ITS) thông qua giám sát lưu lượng tự động đã trở thành yêu cầu cấp thiết. Nhằm khắc phục các hạn chế về chi phí và khả năng mở rộng của phương pháp giám sát truyền thống, nghiên cứu này đề xuất giải pháp ứng dụng Trí tuệ nhân tạo (AI) trong việc nhận diện và phân loại phương tiện từ dữ liệu video thời gian thực. Bằng việc triển khai thuật toán YOLOv11—kiến trúc tiên tiến nhất hiện nay—hệ thống đạt được sự cân bằng tối ưu giữa tốc độ xử lý và độ chính xác. Mô hình được huấn luyện thực tế trên các nhóm phương tiện chủ chốt như xe máy, ô tô, xe buýt và xe tải, cho thấy chỉ số mAP cao trong nhiều điều kiện môi trường phức tạp. Kết quả này đóng vai trò là nền tảng kỹ thuật quan trọng cho các ứng dụng điều tiết tín hiệu tự động và phân tích mật độ giao thông đô thị bền vững.
\end{abstract}

\begin{IEEEkeywords}
Nhận diện phương tiện, Phân loại phương tiện, YOLOv11, Hệ thống giao thông thông minh, Xử lý thời gian thực, Thị giác máy tính.
\end{IEEEkeywords}

\section{GIỚI THIỆU ( INTRODUCTION)}
Trong bối cảnh đô thị hóa diễn ra nhanh chóng, áp lực lên hạ tầng giao thông tại các đô thị lớn ngày càng gia tăng. Tình trạng ùn tắc và mất an toàn giao thông trở thành những vấn đề cấp bách, gây ảnh hưởng trực tiếp đến hiệu quả kinh tế và chất lượng cuộc sống. Theo nhiều nghiên cứu thực tế, các phương pháp giám sát truyền thống thường tốn kém về nhân lực và gặp nhiều hạn chế về khả năng mở rộng quy mô. Do đó, việc xây dựng và phát triển Hệ thống Giao thông Thông minh (ITS) đóng vai trò then chốt trong chiến lược phát triển đô thị bền vững.

Trong những năm gần đây, cùng với sự bùng nổ của trí tuệ nhân tạo và lĩnh vực thị giác máy tính, các phương pháp phân tích video tự động đã được ứng dụng rộng rãi nhằm hỗ trợ quản lý giao thông. Trong đó, bài toán nhận diện và phân loại phương tiện được xem là nhiệm vụ nền tảng nhất, cho phép các cơ quan chức năng chủ động nắm bắt lưu lượng, mật độ dòng xe và đưa ra các quyết định điều tiết tín hiệu đèn hoặc quy hoạch hạ tầng một cách khoa học.

Tuy nhiên, việc xây dựng mô hình nhận diện phương tiện trong môi trường thực tế đặt ra nhiều thách thức lớn. Trước hết, hệ thống đòi hỏi khả năng xử lý thời gian thực (Real-time processing) với độ trễ cực thấp để đảm bảo tính ứng dụng. Bên cạnh đó, dữ liệu video từ camera thường bị ảnh hưởng bởi các yếu tố môi trường như ánh sáng thay đổi, thời tiết xấu và hiện tượng các phương tiện che khuất lẫn nhau (occlusion). Ngoài ra, sự đa dạng về kích thước và đặc trưng hình học giữa các nhóm phương tiện (xe máy, ô tô, xe tải) làm hạn chế hiệu quả của các mô hình phát hiện đối tượng truyền thống.

Xuất phát từ những vấn đề trên, bài báo này tập trung nghiên cứu và xây dựng hệ thống nhận diện, phân loại phương tiện giao thông dựa trên thuật toán YOLOv11—phiên bản tiên tiến nhất trong họ mô hình YOLO. Mục tiêu của nghiên cứu là đánh giá khả năng cân bằng giữa tốc độ và độ chính xác của mô hình trên dữ liệu thực tế, bao gồm các loại xe phổ biến như xe máy, ô tô con, xe buýt và xe tải. Thông qua quá trình thực nghiệm, nghiên cứu hướng đến việc xác định cấu hình tối ưu, đồng thời làm rõ vai trò của các độ đo đánh giá như mAP (mean Average Precision) và FPS trong việc triển khai các ứng dụng giám sát giao thông thông minh.

\section{CÔNG TRÌNH LIÊN QUAN (RELATED WORK)}

Bài toán nhận diện và phân loại phương tiện giao thông (Vehicle Detection and Classification) đã được nghiên cứu rộng rãi trong lĩnh vực thị giác máy tính, đặc biệt là trong các hệ thống giao thông thông minh (ITS), do tác động trực tiếp đến khả năng quản lý lưu lượng và an toàn đường bộ. Các phương pháp truyền thống thường mô hình hóa bài toán này dựa trên việc trích xuất các đặc trưng thủ công như Haar-like, HOG kết hợp với các bộ phân lớp kinh điển như SVM hoặc Decision Tree. Trong đó, các kỹ thuật trừ nền (background subtraction) thường được sử dụng như hướng tiếp cận cơ sở nhờ tính đơn giản và chi phí tính toán thấp.

Tuy nhiên, dữ liệu video giao thông trong thực tế thường chịu ảnh hưởng mạnh bởi các yếu tố ngoại cảnh như sự thay đổi ánh sáng, điều kiện thời tiết khắc nghiệt và hiện tượng che khuất (occlusion), khiến các mô hình truyền thống dễ gặp sai số. Một hướng xử lý phổ biến trước đây là áp dụng các bộ lọc tiền xử lý dữ liệu nhằm cải thiện chất lượng hình ảnh, giúp mô hình học được các ranh giới đối tượng rõ ràng hơn.

Gần đây, các phương pháp dựa trên mạng nơ-ron tích chập (CNN) đã cho hiệu năng vượt trội nhờ khả năng tự động học các biểu diễn đặc trưng phi tuyến và xử lý tốt các tương tác không gian phức tạp. Trong nhóm này, các mô hình phát hiện đối tượng một giai đoạn (one-stage detectors) như dòng thuật toán YOLO (You Only Look Once) được đánh giá cao nhờ cơ chế dự đoán trực tiếp tọa độ hộp bao và xác suất lớp, giúp tối ưu hóa đáng kể tốc độ huấn luyện và xử lý.

Song song đó, các phiên bản cải tiến liên tục được ra đời nhằm cân bằng giữa độ chính xác và tài nguyên tính toán. YOLOv11 là một kiến trúc hiện đại nổi bật, sử dụng các cơ chế chú ý (attention mechanism) theo từng bước để chọn lọc các đặc trưng quan trọng của phương tiện, từ đó nâng cao hiệu năng nhận diện đồng thời hỗ trợ khả năng xử lý thời gian thực. Tuy nhiên, trong bối cảnh giám sát giao thông, việc lựa chọn mô hình cần cân nhắc mục tiêu thực tiễn; do đó các độ đo như độ chính xác trung bình (mAP) và tốc độ khung hình (FPS) thường được ưu tiên hàng đầu để đảm bảo hệ thống hoạt động ổn định và không bỏ sót các phương tiện trong dòng lưu thông mật độ cao.

\section{CƠ SỞ LÝ THUYẾT (Theoretical Background)}

\subsection{Bài toán Nhận diện và Phân loại phương tiện giao thông}
Nhận diện và phân loại phương tiện đề cập đến quá trình xác định sự hiện diện, vị trí thông qua hộp bao (bounding box) và loại của các phương tiện trong một khung hình video. Trong lĩnh vực giao thông thông minh (ITS), bài toán này đóng vai trò nền tảng để thu thập dữ liệu về lưu lượng và mật độ giao thông nhằm tối ưu hóa hạ tầng đô thị. 

Từ góc độ kỹ thuật, đây là bài toán phát hiện đối tượng (Object Detection) đa lớp, nơi mô hình phải đồng thời thực hiện nhiệm vụ xác định tọa độ không gian và gán nhãn cho các thực thể như xe máy, ô tô, xe buýt và xe tải. Đặc điểm của bài toán này đòi hỏi sự chính xác cao trong môi trường động và khả năng phân biệt các đối tượng có kích thước cũng như đặc trưng hình học khác biệt đáng kể.

\subsection{Thuật toán YOLOv11 và bài toán xử lý thời gian thực}
Thách thức lớn nhất trong giám sát giao thông tự động là khả năng xử lý thời gian thực (Real-time processing) trên luồng dữ liệu video có độ phân giải cao nhằm đáp ứng các phản hồi điều khiển tức thời. YOLOv11 (You Only Look Once version 11) được lựa chọn để giải quyết vấn đề này nhờ cơ chế xử lý một giai đoạn (one-stage), cho phép dự đoán trực tiếp các hộp bao và xác suất lớp chỉ qua một lần quét mạng nơ-ron duy nhất.

Cấu trúc của YOLOv11 tập trung vào việc tối ưu hóa sự cân bằng giữa độ chính xác (mAP) và tốc độ khung hình (FPS). Kiến trúc này bao gồm ba thành phần chính:
\begin{itemize}
    \item \textbf{Backbone}: Chịu trách nhiệm trích xuất các đặc trưng đa quy mô từ hình ảnh đầu vào thông qua các lớp tích chập tiên tiến.
    \item \textbf{Neck}: Thực hiện việc kết hợp và tinh chỉnh các đặc trưng để tăng cường khả năng nhận diện đối tượng ở nhiều kích thước khác nhau.
    \item \textbf{Head}: Đưa ra dự đoán cuối cùng về tọa độ hộp bao và xác suất phân loại lớp cho phương tiện.
\end{itemize}

Việc ứng dụng YOLOv11 giúp hệ thống hoạt động ổn định trong các điều kiện môi trường phức tạp như thiếu sáng hoặc mật độ phương tiện dày đặc, đảm bảo dữ liệu đầu ra tin cậy cho các bài toán phân tích mật độ giao thông.
\section{Phương pháp nghiên cứu (Methodology)}
Before you begin to format your paper, first write and save the content as a 
separate text file. Complete all content and organizational editing before 
formatting. Please note sections \ref{AA}--\ref{SCM} below for more information on 
proofreading, spelling and grammar.

Keep your text and graphic files separate until after the text has been 
formatted and styled. Do not number text heads---{\LaTeX} will do that 
for you.

\subsection{Abbreviations and Acronyms}\label{AA}
Define abbreviations and acronyms the first time they are used in the text, 
even after they have been defined in the abstract. Abbreviations such as 
IEEE, SI, MKS, CGS, ac, dc, and rms do not have to be defined. Do not use 
abbreviations in the title or heads unless they are unavoidable.

\subsection{Units}
\begin{itemize}
\item Use either SI (MKS) or CGS as primary units. (SI units are encouraged.) English units may be used as secondary units (in parentheses). An exception would be the use of English units as identifiers in trade, such as ``3.5-inch disk drive''.
\item Avoid combining SI and CGS units, such as current in amperes and magnetic field in oersteds. This often leads to confusion because equations do not balance dimensionally. If you must use mixed units, clearly state the units for each quantity that you use in an equation.
\item Do not mix complete spellings and abbreviations of units: ``Wb/m\textsuperscript{2}'' or ``webers per square meter'', not ``webers/m\textsuperscript{2}''. Spell out units when they appear in text: ``. . . a few henries'', not ``. . . a few H''.
\item Use a zero before decimal points: ``0.25'', not ``.25''. Use ``cm\textsuperscript{3}'', not ``cc''.)
\end{itemize}

\subsection{Equations}
Number equations consecutively. To make your 
equations more compact, you may use the solidus (~/~), the exp function, or 
appropriate exponents. Italicize Roman symbols for quantities and variables, 
but not Greek symbols. Use a long dash rather than a hyphen for a minus 
sign. Punctuate equations with commas or periods when they are part of a 
sentence, as in:
\begin{equation}
a+b=\gamma\label{eq}
\end{equation}

Be sure that the 
symbols in your equation have been defined before or immediately following 
the equation. Use ``\eqref{eq}'', not ``Eq.~\eqref{eq}'' or ``equation \eqref{eq}'', except at 
the beginning of a sentence: ``Equation \eqref{eq} is . . .''

\subsection{\LaTeX-Specific Advice}

Please use ``soft'' (e.g., \verb|\eqref{Eq}|) cross references instead
of ``hard'' references (e.g., \verb|(1)|). That will make it possible
to combine sections, add equations, or change the order of figures or
citations without having to go through the file line by line.

Please don't use the \verb|{eqnarray}| equation environment. Use
\verb|{align}| or \verb|{IEEEeqnarray}| instead. The \verb|{eqnarray}|
environment leaves unsightly spaces around relation symbols.

Please note that the \verb|{subequations}| environment in {\LaTeX}
will increment the main equation counter even when there are no
equation numbers displayed. If you forget that, you might write an
article in which the equation numbers skip from (17) to (20), causing
the copy editors to wonder if you've discovered a new method of
counting.

{\BibTeX} does not work by magic. It doesn't get the bibliographic
data from thin air but from .bib files. If you use {\BibTeX} to produce a
bibliography you must send the .bib files. 

{\LaTeX} can't read your mind. If you assign the same label to a
subsubsection and a table, you might find that Table I has been cross
referenced as Table IV-B3. 

{\LaTeX} does not have precognitive abilities. If you put a
\verb|\label| command before the command that updates the counter it's
supposed to be using, the label will pick up the last counter to be
cross referenced instead. In particular, a \verb|\label| command
should not go before the caption of a figure or a table.

Do not use \verb|\nonumber| inside the \verb|{array}| environment. It
will not stop equation numbers inside \verb|{array}| (there won't be
any anyway) and it might stop a wanted equation number in the
surrounding equation.

\subsection{Some Common Mistakes}\label{SCM}
\begin{itemize}
\item The word ``data'' is plural, not singular.
\item The subscript for the permeability of vacuum $\mu_{0}$, and other common scientific constants, is zero with subscript formatting, not a lowercase letter ``o''.
\item In American English, commas, semicolons, periods, question and exclamation marks are located within quotation marks only when a complete thought or name is cited, such as a title or full quotation. When quotation marks are used, instead of a bold or italic typeface, to highlight a word or phrase, punctuation should appear outside of the quotation marks. A parenthetical phrase or statement at the end of a sentence is punctuated outside of the closing parenthesis (like this). (A parenthetical sentence is punctuated within the parentheses.)
\item A graph within a graph is an ``inset'', not an ``insert''. The word alternatively is preferred to the word ``alternately'' (unless you really mean something that alternates).
\item Do not use the word ``essentially'' to mean ``approximately'' or ``effectively''.
\item In your paper title, if the words ``that uses'' can accurately replace the word ``using'', capitalize the ``u''; if not, keep using lower-cased.
\item Be aware of the different meanings of the homophones ``affect'' and ``effect'', ``complement'' and ``compliment'', ``discreet'' and ``discrete'', ``principal'' and ``principle''.
\item Do not confuse ``imply'' and ``infer''.
\item The prefix ``non'' is not a word; it should be joined to the word it modifies, usually without a hyphen.
\item There is no period after the ``et'' in the Latin abbreviation ``et al.''.
\item The abbreviation ``i.e.'' means ``that is'', and the abbreviation ``e.g.'' means ``for example''.
\end{itemize}
An excellent style manual for science writers is \cite{b7}.

\subsection{Authors and Affiliations}
\textbf{The class file is designed for, but not limited to, six authors.} A 
minimum of one author is required for all conference articles. Author names 
should be listed starting from left to right and then moving down to the 
next line. This is the author sequence that will be used in future citations 
and by indexing services. Names should not be listed in columns nor group by 
affiliation. Please keep your affiliations as succinct as possible (for 
example, do not differentiate among departments of the same organization).

\subsection{Identify the Headings}
Headings, or heads, are organizational devices that guide the reader through 
your paper. There are two types: component heads and text heads.

Component heads identify the different components of your paper and are not 
topically subordinate to each other. Examples include Acknowledgments and 
References and, for these, the correct style to use is ``Heading 5''. Use 
``figure caption'' for your Figure captions, and ``table head'' for your 
table title. Run-in heads, such as ``Abstract'', will require you to apply a 
style (in this case, italic) in addition to the style provided by the drop 
down menu to differentiate the head from the text.

Text heads organize the topics on a relational, hierarchical basis. For 
example, the paper title is the primary text head because all subsequent 
material relates and elaborates on this one topic. If there are two or more 
sub-topics, the next level head (uppercase Roman numerals) should be used 
and, conversely, if there are not at least two sub-topics, then no subheads 
should be introduced.

\subsection{Figures and Tables}
\paragraph{Positioning Figures and Tables} Place figures and tables at the top and 
bottom of columns. Avoid placing them in the middle of columns. Large 
figures and tables may span across both columns. Figure captions should be 
below the figures; table heads should appear above the tables. Insert 
figures and tables after they are cited in the text. Use the abbreviation 
``Fig.~\ref{fig}'', even at the beginning of a sentence.

\begin{table}[htbp]
\caption{Table Type Styles}
\begin{center}
\begin{tabular}{|c|c|c|c|}
\hline
\textbf{Table}&\multicolumn{3}{|c|}{\textbf{Table Column Head}} \\
\cline{2-4} 
\textbf{Head} & \textbf{\textit{Table column subhead}}& \textbf{\textit{Subhead}}& \textbf{\textit{Subhead}} \\
\hline
copy& More table copy$^{\mathrm{a}}$& &  \\
\hline
\multicolumn{4}{l}{$^{\mathrm{a}}$Sample of a Table footnote.}
\end{tabular}
\label{tab1}
\end{center}
\end{table}

\begin{figure}[htbp]
\centerline{\includegraphics{fig1.png}}
\caption{Example of a figure caption.}
\label{fig}
\end{figure}

Figure Labels: Use 8 point Times New Roman for Figure labels. Use words 
rather than symbols or abbreviations when writing Figure axis labels to 
avoid confusing the reader. As an example, write the quantity 
``Magnetization'', or ``Magnetization, M'', not just ``M''. If including 
units in the label, present them within parentheses. Do not label axes only 
with units. In the example, write ``Magnetization (A/m)'' or ``Magnetization 
\{A[m(1)]\}'', not just ``A/m''. Do not label axes with a ratio of 
quantities and units. For example, write ``Temperature (K)'', not 
``Temperature/K''.

\section*{Thực nghiệm và Kết quả (Experiments \& Results)}

The preferred spelling of the word ``acknowledgment'' in America is without 
an ``e'' after the ``g''. Avoid the stilted expression ``one of us (R. B. 
G.) thanks $\ldots$''. Instead, try ``R. B. G. thanks$\ldots$''. Put sponsor 
acknowledgments in the unnumbered footnote on the first page.

\section*{Ứng dụng và Ý nghĩa thực tiễn}

Mô hình có khả năng tích hợp trực tiếp vào các trung tâm điều hành Giao thông Thông minh (ITS) để phân tích lưu lượng và mật độ xe theo thời gian thực. Các chiến lược quản lý đô thị như tự động hóa chu kỳ đèn tín hiệu hoặc phân luồng giao thông dựa trên dữ liệu phân loại phương tiện này mang lại hiệu quả tối ưu hơn so với việc điều phối thủ công truyền thống. Trong bối cảnh xử lý video từ camera giám sát, việc ưu tiên sự cân bằng giữa độ chính xác (mAP) và tốc độ xử lý (FPS) là một lựa chọn kỹ thuật hợp lý nhằm đảm bảo tính kịp thời và chính xác cho công tác quản lý giao thông đô thị.
\section*{Thảo luận và Kết luận}
ABC...

\begin{thebibliography}{00}
\bibitem{b1} G. Eason, B. Noble, and I. N. Sneddon, ``On certain integrals of Lipschitz-Hankel type involving products of Bessel functions,'' Phil. Trans. Roy. Soc. London, vol. A247, pp. 529--551, April 1955.
\bibitem{b2} J. Clerk Maxwell, A Treatise on Electricity and Magnetism, 3rd ed., vol. 2. Oxford: Clarendon, 1892, pp.68--73.
\bibitem{b3} I. S. Jacobs and C. P. Bean, ``Fine particles, thin films and exchange anisotropy,'' in Magnetism, vol. III, G. T. Rado and H. Suhl, Eds. New York: Academic, 1963, pp. 271--350.
\bibitem{b4} K. Elissa, ``Title of paper if known,'' unpublished.
\bibitem{b5} R. Nicole, ``Title of paper with only first word capitalized,'' J. Name Stand. Abbrev., in press.
\bibitem{b6} Y. Yorozu, M. Hirano, K. Oka, and Y. Tagawa, ``Electron spectroscopy studies on magneto-optical media and plastic substrate interface,'' IEEE Transl. J. Magn. Japan, vol. 2, pp. 740--741, August 1987 [Digests 9th Annual Conf. Magnetics Japan, p. 301, 1982].
\bibitem{b7} M. Young, The Technical Writer's Handbook. Mill Valley, CA: University Science, 1989.
\end{thebibliography}
\vspace{12pt}
\color{red}
IEEE conference templates contain guidance text for composing and formatting conference papers. Please ensure that all template text is removed from your conference paper prior to submission to the conference. Failure to remove the template text from your paper may result in your paper not being published.

\end{document}
